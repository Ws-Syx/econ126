\documentclass[11pt,fleqn]{article}
\renewcommand{\baselinestretch}{1.3}
\usepackage{hyperref,harvard,amsmath,amssymb,lscape,graphicx,setspace,cancel,amsthm,color,soul,totcount,graphicx}
%\usepackage{amsfonts,latexsym,eurosym}
\bibliographystyle{aer}

%%%%%%%%%%%%%%%%%%%%%%%%%%%%%%%%%%%%%%%%%%%%%%%%%%%%
% Exam details
\newcommand{\course}{Econ 126: Computational Macroeconomics}
\newcommand{\term}{Winter 2020}
\newcommand{\professor}{Jenkins}
\newcommand{\examdate}{\textbf{Due: 12:30pm, March 20, 2020}}
\newcommand{\examtitle}{\textbf{Final Exam}}
\newcommand{\instructions}{Please read these instructions carefully. There are \total{pnts} points 

\EN
\item \emph{Submission}: You will submit two files. 
	\EN
	\item A .doc, .docx, or .pdf formatted file, with your answers to exam questions. This document must be well organized and easy to read.
	\item An .html version of the Jupyter Notebook file containing all code that you used to generate images and statistics for your exam. Code must be well-organized and thoroughly documented with comments explaining what your code is doing.
	\NE
	
\item \emph{Collaboration}: You may collaborate with others in the class subject to two requirements:
	\EN
	\item At the top of the exam document (the Word or pdf file), you must list the names of everyone in the class that you worked with.
	\item Everyone must prepare their own code. If any two Notebooks are deemed sufficiently similar, then the authors of both Notebooks will receive point reductions up to the entire point value of the exam. Make sure that your comments documenting the code are in your own words.
	\NE
	
\item \emph{Deadline}: Exams will not be accepted after the deadline indicated above.


\NE

}

%%%%%%%%%%%%%%%%%%%%%%%%%%%%%%%%%%%%%%%%%%%%%%%%%%%%
% Other setup commands

\pdfpagewidth 8.5in
\pdfpageheight 11in
\topmargin 0in
\headheight 0in
\headsep 0in
\textheight 9in
\textwidth 6.5in
\oddsidemargin 0in
\evensidemargin 0in
\headheight 0in
\headsep 0in

\newcommand{\D}{\displaystyle}
\newcommand{\E}{\begin{eqnarray*}}
\newcommand{\F}{\end{eqnarray*}}
\newcommand{\EE}{\begin{eqnarray}}
\newcommand{\FF}{\end{eqnarray}}
\newcommand{\IZ}{\begin{itemize}}
\newcommand{\ZI}{\end{itemize}}
\newcommand{\EN}{\begin{enumerate}}
\newcommand{\NE}{\end{enumerate}}
\newcommand{\itemc}{\item[$\circ$]}
\newcommand{\IZdash}{\begin{itemize} \renewcommand{\labelitemi}{-}}
\newcommand{\tn}{\textnormal}

\regtotcounter{pnts}

\newcounter{pnts}
\setcounter{pnts}{0}

\newcommand{\ans}{\item}
\newcommand{\cor}{\item}
\newcommand{\itemt}[1]{\item \textbf{#1}.}
\newcommand{\itemp}[1]{\item (#1 points\addtocounter{pnts}{#1})}

\newenvironment{question}[1]{\item #1\begin{singlespacing}\EN}{\NE\end{singlespacing}}
\newcommand{\bqu}{\begin{question}}
\newcommand{\equ}{\end{question}}

\makeatletter
\newcommand{\shiftleft}{\hspace*{-\@totalleftmargin}}
\newenvironment{fullwidth}
    {\par
     \setlength{\@totalleftmargin}{0pt}%
     \setlength{\linewidth}{\hsize}%
     \list{}{\setlength{\leftmargin}{0pt}}
     \item\relax}
    {\endlist}
%% colored minipage (do not break across pages!)
\newsavebox{\cminibox}
\newlength{\cminilength}
\makeatother

\newcommand{\scenario}[2]{\begin{fullwidth} \textbf{Use the following information to answer the next #1 questions.} #2  \end{fullwidth}}


\onehalfspacing

%%%%%%%%%%%%%%%%%%%%%%%%%%%%%%%%%%%%%%%%%%%%%%%%%%%%%%%%%%%%%%%%%%%%%%%%%%%%%%%%%%%%%%%%%%%%%%%%%%%%
%%%%%%%%%%%%%%%%%%%%%%%%%%%%%%%%%%%%%%%%%%%%%%%%%%%%%%%%%%%%%%%%%%%%%%%%%%%%%%%%%%%%%%%%%%%%%%%%%%%%%

\begin{document}

%%%%%%%%%%%%%%%%%%%%%%%%%%%%%%%%%%%%%%%%%%%%%%%%%%%%%%%%%%%%%%%%%%%%
%%%%%%%%%%%%%%%%%%%%%%%%%%%%%%%%%%%%%%%%%%%%%%%%%%%%%%%%%%%%%%%%%%%%

\begin{table}[ht]
\begin{tabular*}{\textwidth}{l@{\extracolsep{\fill}}r}
\course & %Name:\rule{50mm}{.1pt}

\\
\term: \professor & %ID:\rule{50mm}{.1pt}
\end{tabular*}
\end{table}

\begin{center}
\examtitle\\
\examdate
\end{center}

\

\noindent \textbf{Instructions:} \instructions
%%%%%%%%%%%%%%%%%%%%%%%%%%%%%%%%%%%%%%%%%%%%%%%%%%%%%%%%%%%%%%%%%%%%
%%%%%%%%%%%%%%%%%%%%%%%%%%%%%%%%%%%%%%%%%%%%%%%%%%%%%%%%%%%%%%%%%%%%

\noindent Questions begin on the next page.

\newpage
%\textbf{Questions}


%\section{Factor Prices in a Decentralized RBC Model}

In this problem, you will simulate the dynamic equilibrium of a decentralized RBC model. As usual, the model features an infinitely-lived household that chooses consumption, labor, and capital accumulation to maximize the present value of its lifetime utility. However, instead of assuming that the household produces output goods itself, we'll suppose the existence of a firm sector that pays the household in exchange for capital and labor services. A key product of this modeling approach is that it allows us to model how factor prices -- i.e., the real wage and the capital rental rate -- change over the business cycle.


\subsection{The Model}


\subsubsection{Household Sector}

A representative household lives for an infinite number of periods. The expected present value of lifetime utility to the household from consuming $C_0, C_1, C_2, \ldots $ and working $L_0, L_1, L_2, \ldots $ is denoted by $U_0$:
    \EE
    U_0 & = & E_0\sum_{t = 0}^{\infty} \beta^t\left[ \log (C_t) + \varphi \log (1-L_t) \right],
    \FF	
where $0<\beta<1$ is the household's subjective discount factor and $\varphi$ reflects the relative value that the household places on leisure in the utility function. $E_0$ denotes the expectation with respect to all information available as of date 0.

The household enters period 0 with capital $K_0>0$ and faces the following intertemporal budget constraint:
    \EE
    C_t + K_{t+1} & = & W_t L_t + R_t K_t + (1-\delta)K_t,
    \FF
where $W_t$ is the (real) wage received per unit of labor supplied, $R_t$ is the (real) rental rate per unit of capital, and $\delta$ is the constant rate of capital depreciation.

In period 0, the household solves:
    \EE
    && \max_{C_0,L_0,K_1} \; E_0\sum_{t=0}^{\infty}\beta^t\left[ \log (C_t) + \varphi \log (1-L_t) \right] \nonumber\\
    && \; \; \; \;  \; \; \; \; \text{s.t.} \; \; \; \; C_t + K_{t+1} = W_t L_t + R_t K_t + (1-\delta)K_t
    \FF
which, as usual,  can be written as a choice of $L_0$ and $K_1$ only:
    \EE
    \max_{L_0,K_1} \; E_0\sum_{t=0}^{\infty}\beta^t\left[\log ( W_t L_t + R_t K_t  + (1-\delta)K_t - K_{t+1}) + \varphi \log (1-L_t)\right]
    \FF

\subsubsection{Firm Sector}

Competitive firms produce output $Y_t$ according to the standard Cobb-Douglas production function:
    \EE
    Y_t & = & A_t K_t^{\alpha}L_t^{1-\alpha} \label{eqn:proj1_production}
    \FF
where TFP $A_t$ is stochastic:
\EE
\log A_{t+1} & = & \rho\log A_t + \epsilon_{t+1}  \label{eqn:proj1_tfp}
\FF
Each period, firms take the factor prices $W_t$ and $R_t$ as given and choose $L_t$, $K_t$, $Y_t$ to maximize their profits:
	\EE
    && \max_{Y_t, K_t, L_t} \; Y_t - W_tL_t - R_t K_t \nonumber\\
    && \; \; \; \;  \; \; \; \; \text{s.t.} \; \; \; \; Y_t  =  A_t K_t^{\alpha}L_t^{1-\alpha}
    \FF
The problem can be expressed as an unconstrained problem by substituting $Y_t$ out of the problem:
	\EE
	\max_{K_t, L_t} \; A_t K_t^{\alpha}L_t^{1-\alpha} - W_tL_t - R_t K_t \nonumber\\
	\FF


\subsubsection{Investment}

When the household chooses $K_{t+1}$, it implicitly chooses investment $I_t$ which is defined by:
	\EE
	I_t & = & K_{t+1} - (1-\delta)K_t \label{eqn:proj1_investment}
	\FF

	
\subsubsection{Goods Market Clearing}

In equilibrium, the quantity of goods produced $Y_t$ has to equal the demand for those goods $C_t + I_t$:
	\EE
	Y_t & = & C_t + I_t \label{eqn:proj1_clearing}
	\FF
We call Equation (\ref{eqn:proj1_clearing}) the goods market clearing condition and it represents the aggregate resource constraint for the economy.


\subsubsection{Equilibrium}

The model has 8 endogenous variables: $A_t$, $K_t$, $C_t$, $L_t$, $Y_t$, $I_t$, $W_t$, and $R_t$. Equilibrium is described by:
	\EN
	\item The household's first-order condition for $L_t$
	\item The household's first-order condition for $K_{t+1}$ (the Euler equation)
	\item The firm's first-order condition for $K_t$
	\item The firm's first-order condition for $L_t$
	\item The production function: Equation (\ref{eqn:proj1_production})
	\item The TFP evolution equation: Equation (\ref{eqn:proj1_tfp})
	\item The capital evolution equation: Equation (\ref{eqn:proj1_investment})
	\item The goods  market clearing equation: Equation (\ref{eqn:proj1_clearing})
	\NE

\subsubsection{Calibration}

Assume the following values for the model's parameters:
	
	\
	
	\begin{center}
	\begin{tabular}{ccl}\textbf{Parameter} & \textbf{Value} & \textbf{Description}\\\hline
	$\beta$		& 0.99		& household's subjective discount factor\\
	$\varphi$	& 1.7317 	& household preference parameter\\
	$\alpha$	& 0.35		& Cobb-Douglas production function parameter\\
	$\delta$	& 0.025 	& capital depreciation rate\\
	$\rho$		& 0.75		& autocorrelation of tfp\\
	$\sigma$	& 0.006		& s.d.~of TFP shock\\\hline
	\end{tabular}
	\end{center}

\subsection{Exercises}

\EN
\itemp{4} Solve for the household's first-order conditions for $L_t$ and $K_{t+1}$. Include these equations in your exam document and briefly explain the economic intuition behind them.

\itemp{4} Solve for firm's first-order condition for $K_t$ and $L_t$.  Include these equations in your exam document and briefly explain the economic intuition behind them.

\itemp{6} Use Python to compute the steady state values of $A_t$, $K_t$, $C_t$, $L_t$, $Y_t$, $I_t$, $W_t$, and $R_t$. Include the computed steady state values in your exam document.

\itemp{6} Compute the impulse responses for all of the model's endogenous variables for 41 periods following a one percentage point increase in TFP in period 5. Create a set of clear, easy to read figures that depict the impulse responses and include them in your exam document. Units of plotted quantities should be percent deviations from steady state.\footnote{I.e., multiply the simulated impulse responses by 100.} Describe in words why the behavior of each variable in the simulated impulse responses.

\itemp{6} Compute a 61 period stochastic simulation of model's endogenous variables. Set seed for the random number generator to 2019. Create a set of clear, easy to read figures that depict the stochastic simulation and include them in your presentation. Units of plotted quantities should be percent deviations from steady state. Include the figures in your exam document.

\itemp{4} Which fluctuates more over the business cycle: the real wage or the real rental rate? Explain clearly how you know.

\itemp{10} Make sure all code used to generate results for this problem is well-organized and thoroughly documented.

\NE
\cleardoublepage
\section{Centralized RBC Model with Stochastic Government Consumption}

In this problem, you will simulate the dynamic equilibrium of a centralized RBC model without labor and with stochastic government consumption. As usual, the model features an infinitely-lived household that chooses consumption and capital accumulation to maximize the present value of its lifetime utility. What's new is that we'll assume that there is a government sector that consumes a stochastic quantity of goods each period. A key product of this modeling approach is that it allows us to model how fluctuations in government consumption affect the business cycle.



\subsection{The Model}

\subsubsection{Household Sector}

A representative household lives for an infinite number of periods. The expected present value of lifetime utility to the household from consuming $C_0, C_1, C_2, \ldots $ is denoted by $U_0$:
    \EE
    U_0 & = & E_0\sum_{t = 0}^{\infty} \beta^t \log (C_t),
    \FF	
where $0<\beta<1$ is the household's subjective discount factor. $E_0$ denotes the expectation with respect to all information available as of date 0.

The household enters period 0 with capital $K_0>0$. Production in period $t$ is according to a standard production function that has decreasing returns in capital $K_t$:
    \EE
    F(A_t,K_t) & = & A_t K_t^{\alpha}
    \FF
where TFP $A_t$ is stochastic:
    \EE
    \log A_{t+1} & = & \rho_A\log A_t + \epsilon^A_{t+1} \label{eqn:proj2_tfp}
    \FF
Each period the government collects a lump-sum tax $T_t$ from the household. The household's resource constraint in each period $t$ is therefore:
\EE
C_t + K_{t+1} + T_t & = &  A_t K_{t}^{\alpha}  + (1-\delta)K_t,
\FF
where $\delta$ is the rate of capital depreciation.

In period 0, the household solves:
    \EE
    && \max_{C_0,K_1} \; E_0\sum_{t=0}^{\infty}\beta^t\log (C_t) \nonumber\\
    && \; \; \; \;  \; \; \; \; \text{s.t.} \; \; \; \; C_t + K_{t+1} + T_t = A_tK_t^{\alpha} + (1-\delta)K_t
    \FF
which, as usual,  can be written as a choice of $K_1$ only:
    \EE
    \max_{K_1} \; E_0\sum_{t=0}^{\infty}\beta^t\log ( A_tK_t^{\alpha}  + (1-\delta)K_t - K_{t+1} - T_t)
    \FF

\subsubsection{Government Sector}


Each period the government consumes $G_t$ units of goods. $G_t$ evolves according to the following process:
	\EE
	\log G_{t+1} & = & (1-\rho_G)\log\bar{G}  + \rho_G \log G_{t} + \epsilon^G_{t+1} \label{eqn:proj2_gov}
	\FF
By assumption, the government always runs a balanced budget so:
	\EE
	T_t & = & G_t \label{eqn:proj2_gov_budget}
	\FF


\subsubsection{Investment and Output}

When the household chooses $K_{t+1}$, it implicitly chooses investment $I_t$ which is defined by:
	\EE
	I_t & = & K_{t+1} - (1-\delta)K_t \label{eqn:proj2_investment}
	\FF
and output $Y_t$ which is defined by:
    \EE
    Y_t & = & A_t K_t^{\alpha} \label{eqn:proj2_production}
    \FF

	
\subsubsection{Goods Market Clearing}

In equilibrium, the quantity of goods produced $Y_t$ has to equal the demand for those goods $C_t + I_t + G_t$:
	\EE
	Y_t & = & C_t + I_t + G_t\label{eqn:proj2_clearing}
	\FF
We call Equation (\ref{eqn:proj2_clearing}) the goods market clearing condition and it represents the aggregate resource constraint for the economy.

\subsubsection{Equilibrium}

The model has 7 endogenous variables: $A_t$, $G_t$, $K_t$, $C_t$, $T_t$, $Y_t$, $I_t$. Equilibrium is described by:
	\EN
	\item The household's first-order condition for $K_{t+1}$ (the Euler equation)
	\item The TFP evolution equation: Equation (\ref{eqn:proj2_tfp})
	\item The government consumption evolution equation: Equation (\ref{eqn:proj2_gov})
	\item The government budget constraint: Equation (\ref{eqn:proj2_gov_budget})
	\item The capital evolution equation: Equation (\ref{eqn:proj2_investment})
	\item The production function: Equation (\ref{eqn:proj2_production})
	\item The goods  market clearing equation: Equation (\ref{eqn:proj2_clearing})
	\NE

\subsubsection{Calibration}

Assume the following values for the model's parameters:
	
	\
	
	\begin{center}
	\begin{tabular}{ccl}\textbf{Parameter} & \textbf{Value} & \textbf{Description}\\\hline
	$\beta$		& 0.99		& household's subjective discount factor\\
	$\alpha$	& 0.35		& Cobb-Douglas production function parameter\\
	$\delta$	& 0.025 	& capital depreciation rate\\
	$\rho_A$	& 0.75		& autocorrelation of tfp\\
	$\sigma_A$	& 0.006		& s.d.~of TFP shock\\
	$\bar{G}$	& --		& steady state government consumption\\
	$\rho_G$	& 0.9		& autocorrelation of government consumption\\
	$\sigma_G$	& 0.015		& s.d.~of government consumption shock\\\hline
	\end{tabular}
	\end{center}

\subsection{Exercises}

\EN
\itemp{8} Download two series from FRED\footnote{\href{https://fred.stlouisfed.org/}{https://fred.stlouisfed.org/}}:
	\IZ
	\item[--] Government Consumption Expenditures and Gross Investment (Series ID: GCE)
	\item[--] Gross Domestic Product (Series ID: GCE)
	\ZI
Find the average of the ratio of government consumption to GDP for the US for all dates available.\footnote{Compute the ratio for each date \emph{first}, then compute the average of the ratio.} Report this value in your exam document.


\itemp{4} Solve for the household's first-order condition for $K_{t+1}$. Include this equation in your exam document and be able to explain the intuition behind it.

\itemp{6} Use Python to compute the steady state values of $A_t$, $K_t$, $Y_t$, and $I_t$. You will have to do this manually. Since you don't know $\bar{G}$ yet, you can't use \verb=linearsolve= for this step. Report the computed steady state values of $A_t$, $K_t$, $Y_t$, and $I_t$ in your exam document.

\itemp{6} Use the average ratio of government consumption to GDP for the US to \emph{calibrate} $\bar{G}$:
	\EE
	\bar{G} & = & \bar{Y} \times \Big[\text{Avg.~G-to-Y ratio}\Big]
	\FF
	Then use Python to compute the steady state values of $C_t$ and $T_t$. Report the computed steady state values of $G_t$, $C_t$, and $T_t$ in your exam document.

\itemp{6} Compute the impulse responses for all of the model's endogenous variables for 41 periods following a one percentage point increase in government consumption in period 5.\footnote{Note that like $A_t$, $G_t$ is a state (or predetermined) variable.} Create a set of clear, easy to read figures that depict the impulse responses and include them in your exam document. Units of plotted quantities should be percent deviations from steady state.\footnote{I.e., multiply the simulated impulse responses by 100.} Describe in words why the behavior of each variable in the simulated impulse responses.

\itemp{10} Make sure all code used to generate results for this problem is well-organized and thoroughly documented.
\NE
\newpage
\section{Prescott and Summers} Answer the following.
	
	\EN
	\itemp{6} To what does Prescott solely attribute macroeconomic fluctuations? Explain what this means in your own words.
	\itemp{8} Consider the Euler equation from Prescott's RBC model:
		\EE
		\frac{1}{C_t} & = & \beta E_t\left[\frac{\alpha A_{t+1} K_{t+1}^{\alpha-1}L_{t+1}^{1-\alpha} + 1-\delta}{C_{t+1}}\right]
		\FF
	Explain in words why the left-hand side represents the marginal cost to the household of increasing $K_{t+1}$ and explain why the right-hand side represents the marginal benefit.	
	\itemp{6} Summers' fourth critique of Prescott's work ``is that it ignores the fact that partial breakdowns in the exchange mechanism are almost surely dominant factors in cyclical fluctuations.'' Explain what Summers means by \emph{partial breakdowns in the exchange mechanism} and provide an example from reality that is not explained by Prescott's RBC model.

	
	\NE

\section{Real business cycle and new-Keynesian models} Answer the following.
	
	\EN
	\itemp{5} \emph{Briefly} describe a fundamental similarity between the modeling approaches of real business cycle (RBC) and new-Keynesian (NK) models.
	
	\itemp{5} \emph{Briefly} describe a fundamental difference between the modeling approaches of RBC and NK models.
	\NE


\end{document}
