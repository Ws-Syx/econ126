\section{Factor Prices in a Decentralized RBC Model}

In this problem, you will simulate the dynamic equilibrium of a decentralized RBC model. As usual, the model features an infinitely-lived household that chooses consumption, labor, and capital accumulation to maximize the present value of its lifetime utility. However, instead of assuming that the household produces output goods itself, we'll suppose the existence of a firm sector that pays the household in exchange for capital and labor services. A key product of this modeling approach is that it allows us to model how factor prices -- i.e., the real wage and the capital rental rate -- change over the business cycle.


\subsection{The Model}


\subsubsection{Household Sector}

A representative household lives for an infinite number of periods. The expected present value of lifetime utility to the household from consuming $C_0, C_1, C_2, \ldots $ and working $L_0, L_1, L_2, \ldots $ is denoted by $U_0$:
    \EE
    U_0 & = & E_0\sum_{t = 0}^{\infty} \beta^t\left[ \log (C_t) + \varphi \log (1-L_t) \right],
    \FF	
where $0<\beta<1$ is the household's subjective discount factor and $\varphi$ reflects the relative value that the household places on leisure in the utility function. $E_0$ denotes the expectation with respect to all information available as of date 0.

The household enters period 0 with capital $K_0>0$ and faces the following intertemporal budget constraint:
    \EE
    C_t + K_{t+1} & = & W_t L_t + R_t K_t + (1-\delta)K_t,
    \FF
where $W_t$ is the (real) wage received per unit of labor supplied, $R_t$ is the (real) rental rate per unit of capital, and $\delta$ is the constant rate of capital depreciation.

In period 0, the household solves:
    \EE
    && \max_{C_0,L_0,K_1} \; E_0\sum_{t=0}^{\infty}\beta^t\left[ \log (C_t) + \varphi \log (1-L_t) \right] \nonumber\\
    && \; \; \; \;  \; \; \; \; \text{s.t.} \; \; \; \; C_t + K_{t+1} = W_t L_t + R_t K_t + (1-\delta)K_t
    \FF
which, as usual,  can be written as a choice of $L_0$ and $K_1$ only:
    \EE
    \max_{L_0,K_1} \; E_0\sum_{t=0}^{\infty}\beta^t\left[\log ( W_t L_t + R_t K_t  + (1-\delta)K_t - K_{t+1}) + \varphi \log (1-L_t)\right]
    \FF

\subsubsection{Firm Sector}

Competitive firms produce output $Y_t$ according to the standard Cobb-Douglas production function:
    \EE
    Y_t & = & A_t K_t^{\alpha}L_t^{1-\alpha} \label{eqn:proj1_production}
    \FF
where TFP $A_t$ is stochastic:
\EE
\log A_{t+1} & = & \rho\log A_t + \epsilon_{t+1}  \label{eqn:proj1_tfp}
\FF
Each period, firms take the factor prices $W_t$ and $R_t$ as given and choose $L_t$, $K_t$, $Y_t$ to maximize their profits:
	\EE
    && \max_{Y_t, K_t, L_t} \; Y_t - W_tL_t - R_t K_t \nonumber\\
    && \; \; \; \;  \; \; \; \; \text{s.t.} \; \; \; \; Y_t  =  A_t K_t^{\alpha}L_t^{1-\alpha}
    \FF
The problem can be expressed as an unconstrained problem by substituting $Y_t$ out of the problem:
	\EE
	\max_{K_t, L_t} \; A_t K_t^{\alpha}L_t^{1-\alpha} - W_tL_t - R_t K_t \nonumber\\
	\FF


\subsubsection{Investment}

When the household chooses $K_{t+1}$, it implicitly chooses investment $I_t$ which is defined by:
	\EE
	I_t & = & K_{t+1} - (1-\delta)K_t \label{eqn:proj1_investment}
	\FF

	
\subsubsection{Goods Market Clearing}

In equilibrium, the quantity of goods produced $Y_t$ has to equal the demand for those goods $C_t + I_t$:
	\EE
	Y_t & = & C_t + I_t \label{eqn:proj1_clearing}
	\FF
We call Equation (\ref{eqn:proj1_clearing}) the goods market clearing condition and it represents the aggregate resource constraint for the economy.


\subsubsection{Equilibrium}

The model has 8 endogenous variables: $A_t$, $K_t$, $C_t$, $L_t$, $Y_t$, $I_t$, $W_t$, and $R_t$. Equilibrium is described by:
	\EN
	\item The household's first-order condition for $L_t$
	\item The household's first-order condition for $K_{t+1}$ (the Euler equation)
	\item The firm's first-order condition for $K_t$
	\item The firm's first-order condition for $L_t$
	\item The production function: Equation (\ref{eqn:proj1_production})
	\item The TFP evolution equation: Equation (\ref{eqn:proj1_tfp})
	\item The capital evolution equation: Equation (\ref{eqn:proj1_investment})
	\item The goods  market clearing equation: Equation (\ref{eqn:proj1_clearing})
	\NE

\subsubsection{Calibration}

Assume the following values for the model's parameters:
	
	\
	
	\begin{center}
	\begin{tabular}{ccl}\textbf{Parameter} & \textbf{Value} & \textbf{Description}\\\hline
	$\beta$		& 0.99		& household's subjective discount factor\\
	$\varphi$	& 1.7317 	& household preference parameter\\
	$\alpha$	& 0.35		& Cobb-Douglas production function parameter\\
	$\delta$	& 0.025 	& capital depreciation rate\\
	$\rho$		& 0.75		& autocorrelation of tfp\\
	$\sigma$	& 0.006		& s.d.~of TFP shock\\\hline
	\end{tabular}
	\end{center}

\subsection{Exercises}

\EN
\itemp{4} Solve for the household's first-order conditions for $L_t$ and $K_{t+1}$. Include these equations in your exam document and briefly explain the economic intuition behind them.

\itemp{4} Solve for firm's first-order condition for $K_t$ and $L_t$.  Include these equations in your exam document and briefly explain the economic intuition behind them.

\itemp{6} Use Python to compute the steady state values of $A_t$, $K_t$, $C_t$, $L_t$, $Y_t$, $I_t$, $W_t$, and $R_t$. Include the computed steady state values in your exam document.

\itemp{6} Compute the impulse responses for all of the model's endogenous variables for 41 periods following a one percentage point increase in TFP in period 5. Create a set of clear, easy to read figures that depict the impulse responses and include them in your exam document. Units of plotted quantities should be percent deviations from steady state.\footnote{I.e., multiply the simulated impulse responses by 100.} Describe in words why the behavior of each variable in the simulated impulse responses.

\itemp{6} Compute a 61 period stochastic simulation of model's endogenous variables. Set seed for the random number generator to 2019. Create a set of clear, easy to read figures that depict the stochastic simulation and include them in your presentation. Units of plotted quantities should be percent deviations from steady state. Include the figures in your exam document.

\itemp{4} Which fluctuates more over the business cycle: the real wage or the real rental rate? Explain clearly how you know.

\itemp{10} Make sure all code used to generate results for this problem is well-organized and thoroughly documented.

\NE